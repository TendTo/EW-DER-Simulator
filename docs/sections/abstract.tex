\begin{abstract}
    Con la crescente sensibilità in ambito ambientale e gli avvenimenti geopolitici che si susseguono in questo periodo,
    il mondo della produzione e distribuzione dell'energia elettrica sembra andare incontro ad un processo di decentralizzazione sempre più rapido. \\
    Sono ormai numerose le spinte per rimpiazzare, nel più breve tempo possibile, le fonti energetiche fossili con quelle rinnovabili, quali ad esempio il solare e l'eolico. \\
    Utilizzare questo tipo di risorse introduce delle nuove problematiche che è necessario affrontare.
    La loro produzione dipende da fattori esterni e può variare a causa di eventi difficilmente controllabili, come quelli climatici.
    Inoltre, realtà come privati e piccole imprese hanno sempre più incentivi nel partecipare attivamente nel mercato dell'energia,
    obbligando agenti come i \gls{tso} ad assumere una visione sempre più distribuita della rete elettrica. \\
    Per far fronte a queste problematiche, cercando invece di carpirne il potenziale,
    l'utilizzo di tecnologie distribuite come la blockchain sembra essere una scelta promettente.
    L'utente ottiene anche maggiore controllo sulle informazioni che fornisce, ottenendo inoltre le garanzie di integrità e non ripudio che le tecnologie
    crittografiche combinate con la blockchain sono in grado di fornire. \\
    La relazione in documento descrive l'implementazione di un simulatore che sfrutta i servizi e la blockchain del progetto Energy Web,
    per verificare la fattibilità di una soluzione in grado di gestire i DER e della loro produzione tramite questa tecnologia nell'evento di una richiesta di flessibilità. \\
\end{abstract}
