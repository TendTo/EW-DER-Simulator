\newglossaryentry{did}
{
    name={Decentralized IDentifiers (DID)},
    first={Decentralized IDentifiers (DID)},
    text={DID},
    plural={DIDs},
    description={
            Identificatori che permettono di ottenere un'identità digitale che sia verificabile e decentralizzata.
            Si basano sullo standard stabilito da W3C.
            Nella pratica, si tratta di URIs che puntano ad un DID document.
        }
}
\newglossaryentry{der}
{
    name={Distributed Energy Resource (DER)},
    first={Distributed Energy Resource (DER)},
    text={DER},
    plural={DERs},
    description={
            Risorse fisiche o virtuali in grado di offrire un contributo attivo alla rete elettrica.
            Sono esempi di DER impianti fotovoltaici, turbine eoliche, macchine elettriche, batterie, etc.
        }
}
\newglossaryentry{dsm}
{
    name={Demand Side Management (DSM)},
    first={Demand Side Management (DSM)},
    text={DSM},
    plural={DSMs},
    description={
            Un insieme di azioni volte a gestire in maniera efficiente i consumi di un sito,
            al fine di ridurre i costi sostenuti per l’approvvigionamento di energia elettrica,
            per gli oneri di rete e per gli oneri generali di sistema, incluse le componenti fiscali.
        }
}
\newglossaryentry{irec}
{
    name={International Renewable Energy Certificate (I-REC)},
    first={International Renewable Energy Certificate (I-REC)},
    text={I-REC},
    plural={I-RECs},
    description={
            Standard usato a livello internazionale (in oltre 45 stati) per certificare l'energia proveniente da fonti rinnovabili.
        }
}
\newglossaryentry{poa}
{
    name={Proof of Authority (PoA)},
    first={Proof of Authority (PoA)},
    text={PoA},
    plural={PoAs},
    description={
            Algoritmo di consenso basato su una cerchia ristretta di nodi fidati, i validatori,
            che sono gli unici in grado di aggiungere nuovi blocchi alla blockchain.
        }
}
\newglossaryentry{pos}
{
    name={Proof of Stake (PoS)},
    first={Proof of Stake (PoS)},
    text={PoS},
    plural={PoSs},
    description={
            Algoritmo di consenso che utilizza un processo di elezione pseudo-casuale per selezionare un nodo che agirà da validatore del blocco successivo,
            in base a una combinazione di fattori che possono includere periodo di staking, randomizzazione e fondi di proprietà del nodo.
        }
}
\newglossaryentry{evm}
{
    name={Ethereum Virtual Machine (EVM)},
    first={Ethereum Virtual Machine (EVM)},
    text={EVM},
    plural={EVMs},
    description={
            Macchina virtuale in grado di modificare lo stato della blockchain a cui appartiene.
            Permette l'esecuzione degli smart contract.
        }
}
\newglossaryentry{erc}
{
    name={Ethereum Request for Comment (ERC)},
    first={Ethereum Request for Comment (ERC)},
    text={ERC},
    plural={ERCs},
    description={
            Proposte volte a migliorare Ethereum, generalmente aggiungendo nuove funzionalità e standard.
        }
}
\newglossaryentry{ewns}
{
    name={Energy Web Name Service (EWNS)},
    first={Energy Web Name Service (EWNS)},
    text={EWNS},
    plural={EWNSs},
    description={
            Servizio simile ad un DNS presente su \gls{ewc} che associa ad un indirizzo esadecimale della blockchain un nome di dominio.
        }
}
\newglossaryentry{ewc}
{
    name={Energy Web Chain (EWC)},
    first={Energy Web Chain (EWC)},
    text={EWC},
    plural={EWCs},
    description={
            Blockchain sviluppata da Energy Web. Si tratta di un fork di Ethereum
        }
}
\newglossaryentry{eac}
{
    name={Energy Attribute Certificate (EAC)},
    first={Energy Attribute Certificate (EAC)},
    text={EAC},
    plural={EACs},
    description={
            Certificati emessi come prova di elettricità prodotta da fonti rinnovabili.
            Ogni EAC certifica che 1MWh (o talvolta un KWh) sia stato generato e immesso nella rete da una fonte rinnovabile.
            Alcuni degli standard più comuni sono il Guarantees of Origin (EU), l'I-REC (global) e il REC (US/Canada). \\
            Alcuni termini legati agli EAC sono:
            \begin{itemize}
                \item \textbf{Redeemed, Claimed o Cancelled}: EAC che è stato assegnato o distrutto e che non può essere rivenduto
                \item \textbf{Bundled Certificates}: contratto che vende sia energia che il certificato ad essa associato
                \item \textbf{Unbundled (...)}: contratto che vende solo energia o un EAC, ma non entrambi
            \end{itemize}
        }
}
\newglossaryentry{kms}
{
    name={Key Management System (KMS)},
    first={Key Management System (KMS)},
    text={KMS},
    plural={KMSs},
    description={
            Sistema per la gestione delle chiavi.
        }
}
\newglossaryentry{spf}
{
    name={Single Point of Failure (SPO)},
    first={Single Point of Failure (SPO)},
    text={SPO},
    plural={SPOs},
    description={
            Punto debole di un architettura il cui malfunzionamento può causare anomalie o addirittura la cessazione del servizio da parte del sistema.
        }
}
\newglossaryentry{ipfs}
{
    name={InterPlanetary File System (IPFS)},
    first={InterPlanetary File System (IPFS)},
    text={IPFS},
    plural={IPFSs},
    description={
            Protocollo di comunicazione e una rete peer-to-peer per l'archiviazione e la condivisione di dati in un file system distribuito.
        }
}
\newglossaryentry{dht}
{
    name={Distributed Hash Tables (DHT)},
    first={Distributed Hash Tables (DHT)},
    text={DHT},
    plural={DHTs},
    description={
            Classe di sistemi distribuiti decentralizzati che partizionano l'appartenenza di un set di chiavi tra i nodi partecipanti,
            e possono inoltrare in maniera efficiente i messaggi all'unico proprietario di una determinata chiave.
        }
}
\newglossaryentry{sla}
{
    name={Service-Level Agreement (SLA)},
    first={Service-Level Agreement (SLA)},
    text={SLA},
    plural={SLAs},
    description={
            Strumenti contrattuali attraverso i quali si definiscono le metriche che il fornitore del servizio deve garantire.
        }
}
\newglossaryentry{res}
{
    name={Renewable Energy Sources (RES)},
    first={Renewable Energy Sources (RES)},
    text={RES},
    plural={RESs},
    description={
            Fonti di energia rinnovabile, come impianti fotovoltaici, turbine eoliche, etc.
        }
}
\newglossaryentry{p2p}
{
    name={Peer to Peer (P2P)},
    first={Peer to Peer (P2P)},
    text={P2P},
    plural={P2Ps},
    description={
            Modello di connessione che mette in comunicazione diretta i nodi partecipanti o peer.
        }
}
\newglossaryentry{ew}
{
    name={Energy Web (EW)},
    first={Energy Web (EW)},
    text={EW},
    plural={EWs},
    description={
            Progetto che utilizza tecnologie distribuite per realizzare un ecosistema pensato per il settore energetico.
        }
}
\newglossaryentry{ewdos}
{
    name={Energy Web Decentralized Operating System (EW-DOS)},
    first={Energy Web Decentralized Operating System (EW-DOS)},
    text={EW-DOS},
    plural={EW-DOSs},
    description={
            Schema che riassume il progetto di \gls{ew} e ne descrive l'architettura.
        }
}
\newglossaryentry{dapp}
{
    name={Decentralized Application (DApp)},
    first={Decentralized Application (DApp)},
    text={DApp},
    plural={DApps},
    description={
            Applicazioni che, invece di appoggiarsi su un backend centralizzato tradizionale,
            utilizzano un sistema distribuito, come una blockchain.
        }
}
\newglossaryentry{ewt}
{
    name={Energy Web Token (EWT)},
    first={Energy Web Token (EWT)},
    text={EWT},
    plural={EWTs},
    description={
            Token utilizzato su \gls{ewc}
        }
}
\newglossaryentry{iot}
{
    name={Internet of Things (IoT)},
    first={Internet of Things (IoT)},
    text={IoT},
    plural={IoTs},
    description={
            Rete formata da tutti i piccoli dispositivi in grado di interconnettersi e comunicare,
            generalmente attraverso l'uso di internet.
        }
}
\newglossaryentry{ens}
{
    name={Ethereum Name Service (ENS)},
    first={Ethereum Name Service (ENS)},
    text={ENS},
    plural={ENSs},
    description={
            Servizio simile ad un DNS presente su Ethereum che associa ad un indirizzo esadecimale della blockchain un nome di dominio.
        }
}
\newglossaryentry{ssi}
{
    name={Self Sovereign Identity (SSI)},
    first={Self Sovereign Identity (SSI)},
    text={SSI},
    plural={SSIs},
    description={
            Paradigma che promuove un controllo strettamente personale della propria identità digitale e dei propri dati,
            senza doverli cedere ad un'autorità centrale.
        }
}
\newglossaryentry{iam}
{
    name={Identity and Access Management (IAM)},
    first={Identity and Access Management (IAM)},
    text={IAM},
    plural={IAMs},
    description={
            Criteri in grado di consentire alle organizzazioni di consentire e controllare gli accessi ad applicazioni,
            dati e funzionalità solo ad utenti autorizzati.
        }
}
\newglossaryentry{aggregator}
{
    name={Aggregatore},
    text={aggregatore},
    plural={aggregatori},
    description={
            Un nuovo ruolo che nasce nell'ambito della fornitura di energia e che raggruppa i partecipanti locali (utilizzatori, produttori e \gls{prosumer}). Ha il compito di determinare l'utilizzo dell'energia e può occuparsi della vendita di energia in eccesso prodotta dai \gls{der}.
            L'aggregatore fa da intermediario tra i \gls{prosumer} e i \gls{tso} o i \gls{dso} che si interfacciano con le realtà locali.
        }
}
\newglossaryentry{dso}
{
    name={Distribution System Operator (DSO)},
    first={Distribution System Operator (DSO)},
    text={DSO},
    description={
            Entità che gestiscono (e talvolta possiedono) la rete elettrica a livello regionale o locale.
            Si occupano di principalmente di gestire l'elettricità in ingresso fornita dai \gls{tso} portandola a tensioni adeguate e distribuendola alla rete locale.
        }
}
\newglossaryentry{tso}
{
    name={Transmission System Operator (TSO)},
    first={Transmission System Operator (TSO)},
    text={TSO},
    description={
            Entità responsabile della trasmissione dell'energia elettrica dagli impianti di produzione ai \gls{dso} attraverso la rete elettrica.
            Si occupano di determinare la quantità di elettricità necessaria per la rete in un dato momento e di gestire le riserve energetiche al fine di evitare pericolosi squilibri nella rete.
        }
}
\newglossaryentry{prosumer}
{
    name={Prosumer},
    text={prosumer},
    description={
            Un utente che svolge entrambi i ruoli di consumatore e produttore di energia. Generalmente possiede uno o più \gls{der}.
        }
}
\newglossaryentry{grid-flexibility}
{
    name={Flessibilità della rete},
    text={flessibilità della rete},
    description={
            L'abilità della rete elettrica di mantenere un equilibrio fra l'energia che genera e quella di cui ha bisogno.
            Sia \glspl{tso} che \glspl{dso} si occupano di fornire flessibilità alla rete.
        }
}
\newglossaryentry{oem}
{
    name={Original Equipment Manufacturer (OEM)},
    first={Original Equipment Manufacturer (OEM)},
    text={OEM},
    plural={OEMs},
    description={
            Azienda che manifattura un prodotto o parti di esso per poi venderlo ad un'altra azienda che lo rivenderà ai propri clienti sotto il suo brand.
        }
}
\newglossaryentry{http}
{
    name={Hypertext Transfer Protocol (HTTP)},
    first={Hypertext Transfer Protocol (HTTP)},
    text={HTTP},
    plural={HTTP},
    description={
            Protocollo di comunicazione utilizzato dal web per comunicare fra i client e i server.
            Inizialmente utilizzato solo nella ricezione HTML, ora è stato adottato anche per il trasferimento di dati tramite JSON e gli endpoint REST.
        }
}
\newglossaryentry{vc}
{
    name={Verifiable Credentials (VC)},
    first={Verifiable Credentials (VC)},
    text={VC},
    plural={VCs},
    description={
            Insieme di affermazioni effettuate a proposito di un soggetto, generalmente un \gls{did}, da un autorità riconosciuta.
            Sono utilizzate per permettere a chiunque di verificare le proprietà di un soggetto e chi le ha certificate.
        }
}
\newglossaryentry{lan}
{
    name={Local Area Network (LAN)},
    first={Local Area Network (LAN)},
    text={LAN},
    plural={LANs},
    description={
            Identifica una rete costituita da computer collegati tra loro, dalle interconnessioni e dalle periferiche condivise in un ambito fisico delimitato in una breve distanza.
            proprio grazie alle dimensioni contenute, assicurano un basso tempo di trasmissione.
        }
}
\newglossaryentry{api}
{
    name={Application Programming Interfaces (API)},
    first={Application Programming Interfaces (API)},
    text={API},
    plural={APIs},
    description={
            Interfacce che permettono alle applicazioni di comunicare e scambiare dati con altre applicazioni.
            Questo viene realizzato tramite un insieme di definizioni e protocolli condivisi standardizzati.
        }
}
\newglossaryentry{rest}
{
    name={REpresentational State Transfer (REST)},
    first={REpresentational State Transfer (REST)},
    text={REST},
    plural={REST},
    description={
            \gls{api} per la trasmissione di dati su \gls{http} stateless.
            Utilizza una combinazione di \gls{url} e verbi \gls{http} per identificare
            rispettivamente la risorsa (o le risorse) e l'azione da compiere.
        }
}
\newglossaryentry{url}
{
    name={Uniform Resource Locator (URL)},
    first={Uniform Resource Locator (URL)},
    text={URL},
    plural={URLs},
    description={
            Sequenza di caratteri che identifica univocamente l'indirizzo di una risorsa su una rete di computer.
        }
}
\newglossaryentry{jwt}
{
    name={JSON Web Token (JWT)},
    first={JSON Web Token (JWT)},
    text={JWT},
    plural={JWTs},
    description={
            Token di autenticazione utilizzato per identificare un utente che ha effettuato un accesso ad una risorsa.
            Il token ha una durata prestabilita, dopo la quale deve essere rinnovato.
        }
}
\newglossaryentry{bom}
{
    name={Bill Of Materials (BOM)},
    first={Bill Of Materials (BOM)},
    text={BOM},
    plural={BOMs},
    description={
            Lista di software, framework e librerie utilizzate per realizzare una applicazione.
        }
}
\newglossaryentry{sdk}
{
    name={Software Developer Kit (SDK)},
    first={Software Developer Kit (SDK)},
    text={SDK},
    plural={SDKs},
    description={
            Insieme di strumenti che consente lo sviluppo di software o firmware per una specifica piattaforma.
        }
}
\newglossaryentry{blockchain}
{
    name={Blockchain},
    text={blockchain},
    description={
            La blockchain è una struttura dati pubblica e immutabile.
            È definita come un registro digitale le cui voci sono raggruppate in "blocchi", concatenati in ordine cronologico, e la cui integrità è garantita dall'uso della crittografia.
            Il suo contenuto viene aggiornato tramite un processo normato, in grado di verificare l'attendibilità e validità dei nuovi blocchi.
        }
}
\newglossaryentry{smart-contract}
{
    name={Smart contract},
    text={smart contract},
    plural={smart contracts},
    description={
            Uno smart contract è un programma in grado di essere eseguito sulla blockchain.
            Come ogni altra cosa in ambito blockchain, una volta entrato a far parte dei quest'ultima
            il codice non può più essere modificato. \\
            Tuttavia gli smart contract hanno la possibilità di possedere uno stato interno che può essere
            alterato con l'invocazione di uno dei suoi metodi da parte di un \gls{eoa}.
        }
}
\newglossaryentry{agreement}
{
    name={Agreement},
    text={agreement},
    plural={agreements},
    description={
            Accordo fra l'\gls{aggregator} e il \gls{prosumer} che sancisce i termini della loro collaborazione.
            In questo tipo di dato vengono infatti definiti la fonte energetica che il \gls{prosumer} utilizza,
            la produzione elettrica, il prezzo pattuito e la flessibilità che è in grado di fornire.
        }
}\newglossaryentry{rpc-api}
{
    name={JSON-RPC API},
    text={JSON-RPC API},
    description={
            Per poter interagire con la blockchain, è necessario passare attraverso un nodo che ne faccia parte.
            Poiché possederne e mantenerne uno potrebbe essere uno sforzo proibitivo per l'utente medio, esistono dei nodi pubblici
            che mettono a disposizione un'interfaccia JSON-RPC, trasmettendo poi le richieste alla rete di nodi. \\
            Questo permette ai client di invocare metodi di uno \gls{smart-contract} o di conoscere lo stato della blockchain tramite una semplice chiamata HTTP, HTTPs o websocket.
        }
}
\newglossaryentry{ssr}
{
    name={Server-side rendering (SSR)},
    first={Server-side rendering (SSR)},
    text={SSR},
    description={
            Contrariamente ad un sisto statico, che si limita ad inviare al client un file HTML preesistente in risposta alla sua richiesta,
            il server side rendering, come dice il nome, prevede che la pagina HTML sia generata nel momento della richiesta, avendo quindi la possibilità di integrare informazioni esterne come i dati di un database. \\
        }
}
\newglossaryentry{metamask}
{
    name={MetaMask},
    description={
            MetaMask è un'estensione o un plugin per browser web che permette agli utenti di interagire facilmente con le dApp del blockchain di Ethereum.
            Questo è possibile, perché MetaMask funge da ponte tra dApp e browser web, facilitandone l'utilizzo e il divertimento.}
}
\newglossaryentry{event-log}
{
    name={Event Log},
    text={event log},
    description={
            Output che uno \gls{smart-contract} può produrre con la codeword \texttt{emit}.
            L'output prodotto viene aggiunto in maniera permanente alla blockchain e può essere consultato da chiunque. \\
            È possibile inserire un numero arbitrario di parametri che saranno visibili nel log, e fino a tre parametri indicizzati,
            che possono essere usati per filtrare solo i log pertinenti. \\
            Possono rappresentare un'alternativa più economica per lo storage dei dati.
        }
}
\newglossaryentry{baseline}
{
    name={Baseline dell'aggregatore},
    text={baseline},
    description={
            Apporto energetico che i \gls{der} aggregati sono in grado di produrre a regime.
            Sebbene il valore reale potrebbe essere soggetto a variazioni, si presuppone che le oscillazioni siano centrate sulla baseline. \\
            Nal caso di una richiesta di flessibilità, la variazione è calcolata a partire dalla baseline.
        }
}
\newglossaryentry{eoa}
{
    name={Externally Owned Account (EOA)},
    first={Externally Owned Account (EOA)},
    text={EOA},
    plural={EOAs},
    description={
            Sono account della blockchain identificati da un indirizzo pubblico associato ad una chiave privata.
            Chiunque sia in possesso della chiave privata è in grado di firmare transazioni a nome dell'account. \\
            Per invocare un metodo di uno \gls{smart-contract}, è necessario che la transazione sia stata originata da un account EOA.
        }
}